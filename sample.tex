\documentclass[12pt]{book}
\usepackage{amsmath,amssymb,amsthm}
\usepackage{enumitem}
\usepackage[utf8]{vietnam}
\usepackage{tgtermes}
\usepackage{lipsum}
\usepackage{tikz}
\usetikzlibrary{tikzmark,arrows.meta,fit}
\usepackage{centernot}
\usepackage{geometry}
\usepackage{setspace}
\onehalfspacing

\geometry{a4paper, total={6in, 8in}}
\everymath{\displaystyle}
\theoremstyle{definition}
\newtheorem{definition}{Definition}[section]
\newtheorem*{remark}{Remark}
\usepackage{lecture-design}

\usepackage{hyperref}

\renewcommand{\TypesetSolutionHint}[1]{[Xem lời giải trang: #1]}
\begin{document}
\chapter{Chuỗi số}
\section{Đại cương về chuỗi số}
Khái niệm chuỗi số:
\[
  \overbrace{\sum_{n=1}^{\infty}a_{\tikzmark{shtq}n}=\underbrace{a_{1}+a_{2}+a_{3}+\cdots+a_{n}}_{{\displaystyle
      S_{n}=\sum_{k=1}^{n}a_{k}}\textup{ tổng riêng thứ
  }n}+\cdots}^{\textup{chuỗi số}}
\]
\begin{tikzpicture}[overlay, remember picture]
  \draw[->, thick]
  (pic cs:shtq) ++ (0, -2pt) -- ++(0,-1.5)
  node[below] {Số hạng tổng quát};
\end{tikzpicture}
% \begin{definition}
%   Hội tụ/phân kỳ:
%   \begin{itemize}
%     \item Chuỗi hội tụ:
%       $\lim_{n\to+\infty}S_{n}=S\in\mathbb{R}\implies
%       S=\sum_{n=1}^{\infty}a_{n}.$
%     \item Chuỗi phân kỳ: $\lim_{n\to+\infty}S_{n}=\infty$ hoặc
%       $\centernot{\exists}\lim_{n\to+\infty}S_{n}$
%   \end{itemize}
% \end{definition}

\begin{problem}
  BAC
\end{problem}

\begin{theorem}
  Một số tính chất:
  \begin{itemize}
    \item Chuỗi $\sum_{n=1}^{\infty}a_{n}$ và $\sum_{n=N_{0}}^{+\infty}a_{n}$
      cùng hội tụ/phân kỳ.
    \item Tuyến tính $\sum_{n=1}^{+\infty}\left(\alpha a_{n}+\beta
      b_{n}\right)=\alpha.\left(\sum_{n=1}^{+\infty}a_{n}\right)+\beta\cdot\left(\sum_{n=1}^{+\infty}b_{n}\right)$
    \item $\sum_{n=1}^{+\infty}a_{n}$ hội tụ $\substack{\implies\\
        \centernot{\impliedby}
      }
      \lim_{n\to+\infty}a_{n}=0$
  \end{itemize}
\end{theorem}

\begin{problem}
  Xét hội tụ, phân kỳ chuỗi (TCSS)
  \[
    \sum_{n=1}^{+\infty}\sin\left(\frac{1}{n^{k}}\right), k>0
  \]

  \begin{solution}
    \begin{enumerate}
      \item $\frac{1}{n^{2}}<\frac{\pi}{2}, \frac{1}{n^{2}}$ giảm, chuỗi dương
      \item $ \sin\frac{1}{n^{2}}\sim\frac{1}{n^{2}},n\to+\infty $
    \end{enumerate}
  \end{solution}
\end{problem}

\begin{problem}[Phần 3]%
  Xét hội t\d{u}, phân k\`{y} chuỗi (cho các tiêu chuẩn còn l\d{a}i)

  \begin{enumerate}
    \item $\sum_{n=1}^{+\infty}\frac{n^{2}}{\alpha^{n}}$ $\alpha>1$ (Answer: HT)
      \begin{solution}
        asdasd
      \end{solution}
    \item $\sum_{n=1}^{\infty}\frac{\left(3n+4\right)!}{n^{2}5^{n}}$,
      \begin{solution}
        \begin{enumerate}
          \item Đặt $a_{n}=\frac{n^{2}}{\alpha^{n}}$,
            $a_{n+1}=\frac{\left(n+1\right)^{2}}{\alpha^{n+1}}=\frac{n^{2}+2n+1}{\alpha\cdot\alpha^{n}}$
          \item Có
            $\frac{a_{n+1}}{a_{n}}=\frac{n^{2}+2n+1}{\alpha\cdot\alpha^{n}}\cdot\frac{\alpha^{n}}{n^{2}}=\frac{1}{\alpha}\cdot\frac{n^{2}+2n+1}{\alpha}\to\frac{1}{\alpha}$
            khi $n\to\infty$
          \item Nếu $\alpha>1$, chuỗi HT (D\textbackslash Alembert)
          \item Nếu $\alpha<1$ chuỗi PK
          \item Nếu $\alpha=1$, $a_{n}=n^{2}$, $\sum_{n=1}^{+\infty}a_{n}$ PK
        \end{enumerate}
      \end{solution}
    \item $\sum_{n=1}^{+\infty}\frac{an+b}{cn+d}$ $a,b,c,d>0$,
  \end{enumerate}
\end{problem}

\section{Section 2}
content:
\begin{problem}

  x

  \begin{equation}
    \int_{1}^{\infty} f(x)dx
  \end{equation}
  \begin{solution}
    solution

  \end{solution}
\end{problem}

\begin{theorem}[So sánh 1 -- Direct comparison]
  \label{thm:tc-so-sanh-1} $0\le a_{n}\le b_{n}$ $\forall n\ge
  N_{0}\in\mathbb{N}$,
  \begin{enumerate}
    \item (B) HT $\implies$(A) HT
    \item (A) PK $\implies$ (B) PK
  \end{enumerate}
  \begin{proof}
    \begin{itemize}
      \item K mất tổng quát, $N_{0}=1$, $a_{n}\le b_{n}$ với $\forall n\ge1$,
        $\sum_{n=1}^{+\infty}b_{n}$ HT, thì $\sum_{n=1}^{+\infty}a_{n}$
      \item Đặt:
        \begin{align*}
          A_{n} & =a_{1}+a_{2}+a_{3}+\cdots+a_{n},\\
          B_{n} & =b_{1}+b_{2}+b_{3}+\cdots+b_{n},
        \end{align*}
      \item Đặt $B_{n}=\sum_{n=1}^{N}b_{n}$, $\lim_{n\to\infty}B_{n}=B$. Vì
        $b_{n}$ dương, nên $B_{n}$ là dãy tăng, do đó $B_{n}\le B$.Vì
        $a_{n}\le b_{n}$
        nên $A_{n}\le B_{n}\le B$.
      \item Chuỗi dương nên $A_{n}$ tăng, mà $A_{n}\le B$ do đó
        $\exists\lim_{n\to\infty}A_{n}\in\mathbb{R}$.
      \item Ngược lại, giả sử $A_{n}$ phân kỳ nhưng $B_{n}\to B$, khi đó $A_{n}$
        tăng và bị chặn $\implies$ luôn hội tụ $\implies$ mâu thuẫn
      \item Do đó nếu $A_{n}$ phân kỳ thì $B_{n}$ phân kỳ. \qedhere
    \end{itemize}
  \end{proof}
\end{theorem}

\begin{theorem}
  \begin{proof}
    \begin{itemize}
      \item K mất tổng quát, $N_{0}=1$, $a_{n}\le b_{n}$ với $\forall n\ge1$,
        $\sum_{n=1}^{+\infty}b_{n}$ HT, thì $\sum_{n=1}^{+\infty}a_{n}$
      \item Đặt:
        \begin{align*}
          A_{n} & =a_{1}+a_{2}+a_{3}+\cdots+a_{n},\\
          B_{n} & =b_{1}+b_{2}+b_{3}+\cdots+b_{n},
        \end{align*}
      \item Đặt $B_{n}=\sum_{n=1}^{N}b_{n}$, $\lim_{n\to\infty}B_{n}=B$. Vì
        $b_{n}$ dương, nên $B_{n}$ là dãy tăng, do đó $B_{n}\le B$.Vì
        $a_{n}\le b_{n}$
        nên $A_{n}\le B_{n}\le B$.
      \item Chuỗi dương nên $A_{n}$ tăng, mà $A_{n}\le B$ do đó
        $\exists\lim_{n\to\infty}A_{n}\in\mathbb{R}$.
      \item Ngược lại, giả sử $A_{n}$ phân kỳ nhưng $B_{n}\to B$, khi đó $A_{n}$
        tăng và bị chặn $\implies$ luôn hội tụ $\implies$ mâu thuẫn
      \item Do đó nếu $A_{n}$ phân kỳ thì $B_{n}$ phân kỳ. \qedhere
    \end{itemize}
  \end{proof}
\end{theorem}

\chapter{Chứng minh}
%
\ShowProofs
%  \GetArr{problem-body}{2}
\chapter{Giải bài tập}

\ShowSolutions
\end{document}
